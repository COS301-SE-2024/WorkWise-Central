\documentclass{article}

\title{%
	Requirements specifications \\
	\large WorkWise Connect}
\author{Boolean Hooligans}

	
\begin{document}
	
	\maketitle
	\newpage
	
	\section*{Introduction}	
	
	\newpage
	
	\section*{User Stories}
	
	\newpage
	
	\section*{Function requirements}
	\subsection*{Authorization and Authentication subsystem}	
	\subsection*{Account Management subsystem}	
	\subsection*{Job Management subsystem}	

	\subsubsection*{Manager Job management}	
	\subsubsection*{Employee Job management}	
	\subsection*{Inventory management subsystem}
	The inventory management subsystem serves as the backbone for tracking, updating, and managing the resources essential for the seamless delivery of services provided by the company. It ensures that adequate inventory levels are maintained, minimizing delays and ensuring smooth operations.

	\subsubsection*{Manager Inventory management}	
	The Manager Inventory Management feature empowers managers with comprehensive control over the company's inventory. Managers should have the capability to perform a range of actions including:
	Adding Inventory Items: Managers should be able to effortlessly add new inventory items into the system, specifying details such as item name, category, quantity, and relevant attributes.
	Updating Inventory Information: Managers should have the ability to update existing inventory records, modifying details like quantity on hand, location, and associated costs.
	Deleting Inventory Items: In cases where certain inventory items are no longer relevant or available, managers should be able to securely delete them from the system.
	Viewing Detailed Reports: The system should provide managers with insightful reports regarding inventory usage and status. These reports should encompass vital metrics such as inventory turnover rates, stock-out occurrences, and trends over time, aiding in informed decision-making.
	\subsubsection*{Employee Inventory management}	
	The Employee Inventory Management functionality empowers frontline employees with the necessary tools to efficiently interact with the inventory system. Key features include:
	Viewing Inventory Items: Employees should have access to a comprehensive list of inventory items, categorized and easily searchable, along with pertinent details such as item descriptions and current availability.
	Updating Inventory Status: Employees should be able to update the status of inventory items in real-time, marking them as in use, available, in need of replacement, or any other relevant status. This ensures accurate tracking of inventory utilization and facilitates timely replenishment.
	Requesting Additional Inventory Items: Employees should have the capability to submit requests for additional inventory items as per operational requirements. These requests should be seamlessly integrated into the inventory management workflow, triggering appropriate actions by managers.
	\subsection*{Real-time Communication and notification subsystem}	
	The Real-time Communication and Notification subsystem serve as a vital conduit for facilitating seamless communication among managers, employees, and clients, ensuring timely updates and fostering transparency in operations.
	Push Notifications and SMS Alerts: The system should support the delivery of push notifications and SMS alerts to relevant stakeholders, including managers, employees, and clients. These notifications should encompass critical updates such as job assignments, inventory changes, and urgent announcements, ensuring prompt attention and action.
	Location Tracking: Real-time location tracking capabilities should be integrated into the system, enabling stakeholders to track the progress of jobs and deliveries in real-time. This feature enhances operational visibility and enables proactive decision-making based on the current status of activities.
	Chat and Messaging: The system should provide chat and messaging functionality for seamless communication among team members, enabling quick exchanges of information, updates, and clarifications. This feature fosters collaboration and teamwork, enhancing operational efficiency and service quality.

	\subsection*{Reporting subsystem}	
	The Reporting subsystem empowers managers with actionable insights derived from comprehensive reports, facilitating data-driven decision-making and performance optimization across various facets of business operations.
	Customizable Reports: Managers should have the flexibility to generate customized reports tailored to specific parameters and time periods. These reports should cover diverse aspects such as job progress, inventory status, employee performance, and client satisfaction, providing a holistic view of business performance.
	Insightful Analytics: In addition to standard reports, the system should offer advanced analytical capabilities, including trend analysis, forecasting, and performance benchmarks. These analytics empower managers to identify patterns, anticipate future needs, and proactively address challenges, driving continuous improvement.
	Automated Report Generation: The system should support automated report generation and distribution, streamlining the reporting process and ensuring timely access to critical information. Managers should be able to schedule report generation at regular intervals and receive reports via email or other channels, reducing manual effort and enhancing efficiency.

	\subsection*{Feedback subsystem}	
	The Feedback subsystem serves as a valuable mechanism for soliciting and acting upon feedback from clients, fostering a culture of continuous improvement and enhancing overall service quality.
	Client Feedback Mechanism: The system should provide clients with intuitive channels for submitting feedback, including options for rating the service received, providing detailed comments, and suggesting improvements. This feedback mechanism should be seamlessly integrated into the client interaction process, ensuring convenience and accessibility.
	Manager Review and Action: Feedback submitted by clients should be promptly accessible to managers for review and action. Managers should have the capability to analyze feedback trends, identify recurring issues or areas for improvement, and take proactive measures to address them, thereby enhancing client satisfaction and loyalty.
	Performance Metrics: The system should capture and analyze key performance metrics derived from client feedback, such as satisfaction scores, response times, and service ratings. These metrics serve as valuable indicators of service quality and can be used to drive targeted improvements and training initiatives.

	\subsection*{Job Scheduling subsystem}
	The Job Scheduling subsystem plays a pivotal role in optimizing resource allocation, streamlining operations, and ensuring timely delivery of services to clients.
	Automated Job Assignment: The system should support automated job assignment based on predefined criteria such as employee availability, skill set, and proximity to the job location. This feature minimizes manual intervention, reduces errors, and ensures efficient utilization of resources.
	Real-time Job Updates: Managers and employees should receive real-time updates regarding job assignments, schedule changes, and client requests. These updates should be delivered through push notifications, SMS alerts, or other communication channels, enabling stakeholders to stay informed and responsive.
	Optimization Algorithms: The system should leverage optimization algorithms to intelligently schedule jobs, considering factors such as travel time, resource availability, and client preferences. These algorithms help in maximizing operational efficiency, minimizing delays, and enhancing overall service quality.

	\subsection*{Appointment Scheduling and Tracking subsystem}
	The Appointment Scheduling and Tracking subsystem provides employees/managers with a user-friendly platform for booking services, managing appointments, and tracking service providers in real-time.


	
	\newpage
	
	\section*{Service contract}	
	
	\newpage
	
	\section*{Class diagram}
	
	\newpage
	
	\section*{Architectural requirements}
	
	\subsection*{Quality requirements}	
	\subsection*{Architectural Patterns}	
	\subsection*{Design Patterns}	
	\subsection*{Constraints}	
	
	\newpage
	\section*{Technology Requirements}
	
	
\end{document}